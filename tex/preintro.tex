% ---
% Inserir a ficha bibliografica
% ---

% Isto é um exemplo de Ficha Catalográfica, ou ``Dados internacionais de
% catalogação-na-publicação''. Você pode utilizar este modelo como referência. 
% Porém, provavelmente a biblioteca da sua universidade lhe fornecerá um PDF
% com a ficha catalográfica definitiva após a defesa do trabalho. Quando estiver
% com o documento, salve-o como PDF no diretório do seu projeto e substitua todo
% o conteúdo de implementação deste arquivo pelo comando abaixo:
%
% \begin{fichacatalografica}
%     \includepdf{fig_ficha_catalografica.pdf}
% \end{fichacatalografica}
\imprimirfichacatalografica
\begin{comment}
\begin{fichacatalografica}
	\vspace*{\fill}					% Posição vertical
	\hrule							% Linha horizontal
	\begin{center}					% Minipage Centralizado
	\begin{minipage}[c]{12.5cm}		% Largura
	
	\imprimirautor
	
	\hspace{0.5cm} \imprimirtitulo  / \imprimirautor. --
	\imprimirlocal, \imprimirdata-
	
	\hspace{0.5cm} \pageref{LastPage} p. : il. (algumas color.) ; 30 cm.\\
	
	\hspace{0.5cm} \imprimirorientadorRotulo~\imprimirorientador\\
	
	\hspace{0.5cm}
	\parbox[t]{\textwidth}{\imprimirtipotrabalho~--~\imprimirinstituicao,
	\imprimirdata.}\\
	
	\hspace{0.5cm}
		1. Palavra-chave1.
		2. Palavra-chave2.
		I. Orientador.
		II. Universidade xxx.
		III. Faculdade de xxx.
		IV. Título\\ 			
	
	\hspace{8.75cm} CDU 02:141:005.7\\
	
	\end{minipage}
	\end{center}
	\hrule
\end{fichacatalografica}
\end{comment}
% ---

% ---
% Inserir errata
%

% ---

% ---
% Inserir folha de aprovação
% ---

% Isto é um exemplo de Folha de aprovação, elemento obrigatório da NBR
% 14724/2011 (seção 4.2.1.3). Você pode utilizar este modelo até a aprovação
% do trabalho. Após isso, substitua todo o conteúdo deste arquivo por uma
% imagem da página assinada pela banca com o comando abaixo:
%
% \includepdf{folhadeaprovacao_final.pdf}
%
\begin{folhadeaprovacao}

  \begin{center}
    {\ABNTEXchapterfont\large\imprimirautor}

    \vspace*{\fill}\vspace*{\fill}
    \begin{center}
      \ABNTEXchapterfont\bfseries\Large\imprimirtitulo
    \end{center}
    \vspace*{\fill}
    
    \hspace{.45\textwidth}
    \begin{minipage}{.5\textwidth}
        \imprimirpreambulo
    \end{minipage}%
    \vspace*{\fill}
   \end{center}
        
   Trabalho aprovado. \imprimirlocal, XX de novembro de 2015:

   \assinatura{\textbf{\imprimirorientador} \\ Orientador} 
   \assinatura{\textbf{Professor} \\ Convidado 1}
   \assinatura{\textbf{Professor} \\ Convidado 2}
   %\assinatura{\textbf{Professor} \\ Convidado 3}
   %\assinatura{\textbf{Professor} \\ Convidado 4}
      
   \begin{center}
    \vspace*{0.5cm}
    {\large\imprimirlocal}
    \par
    {\large\imprimirdata}
    \vspace*{1cm}
  \end{center}
  
\end{folhadeaprovacao}
% ---

% ---
% Dedicatória
% ---
\begin{dedicatoria}
   \vspace*{\fill}
   \centering
   \noindent
    \textit{ Dedico este trabalho à Dona Mirian, minha mãe.} \vspace*{\fill}
\end{dedicatoria}
% ---

% ---
% Agradecimentos
% ---
\begin{agradecimentos}
%To be done.
\begin{comment}

Agradeço a Heloisa, sua ajuda e motivação fez com que esse trabalho acontecece.

Aos meus amigos, Lucas Stéfano, João Gabriel Reis, Petrus, pelo apoio e ajuda com revisões.

Depois de quase nove anos de curso, agradeço especialmente à paciência da minha família, e em especial da minha mãe. 

Ao povo brasileiro que pagam pela universidade pública. Espero que meu trabalho e carreira possam refletir numa nação melhor para seu povo.

E a todas as outras pessoas, que não
\end{comment}
\end{agradecimentos}
% ---

% ---
% Epígrafe
% ---
\begin{epigrafe}
    \vspace*{\fill}
	\begin{flushright}
%		\textit{``epigrafe\\
%epigrafe\\
%epigrafe.'' \\
 %       } Autor\\
	\end{flushright}
\end{epigrafe}
% ---

% ---
% RESUMOS
% ---

% resumo em português
\setlength{\absparsep}{18pt} % ajusta o espaçamento dos parágrafos do resumo
\begin{resumo}
    Este trabalho relata o processo de desenvolvimento de um software de testes funcional e estrutural para validar o funcionamento e detectar falhas em placas eletrônicas no processo produtivo assim como nos retornos de manutenção. O desenvolvimento foi feito à partir de uma jiga de testes com interface serial com o dispositivo, testes no circuito por pontas de prova e de potência de sinal RF. Focando em modularidade e concorrência, o programa foi baseado no Modelo de Atores e implementado no ambiente LabView. %Este trabalho também contém uma revisão do estado da arte em testes e diagnósticos sistêmicos de alta qualidade para circuitos eletrônicos, como também as práticas habituais e desafios na atualidade. Técnicas especializadas e padrões industriais para teste de placas complexas são introduzidas. 

 \textbf{Palavras-chaves}: teste funcional. teste estrutural. modelo de atores.
\end{resumo}

% resumo em inglês
\begin{resumo}[Abstract]
 \begin{otherlanguage*}{english}
   
   This is the english abstract. 

   \vspace{\onelineskip}
 
   \noindent 
   \textbf{Key-words}: funcional test. structural test. actor model. 
 \end{otherlanguage*}
\end{resumo}

% ---
% inserir lista de ilustrações
% ---
\pdfbookmark[0]{\listfigurename}{lof}
\listoffigures*
\cleardoublepage
% ---

% ---
% inserir lista de tabelas
% ---
\pdfbookmark[0]{\listtablename}{lot}
\listoftables*
\cleardoublepage
% ---

% ---
% inserir lista de abreviaturas e siglas
% ---
\begin{siglas}
    \item [AOI] Automated Optical Inspection
    \item [ASIC] Application-Specific Integrated Circuit 
    \item [ATS] \textit{Automated Test Systems}
    \item [AXI] \textit{Automated X-ray Inspection}
    \item [BERT] \textit{Bit-error Rate}
    \item [BISD] \textit{Build-in Self Diagnosis}
    \item [BIST] \textit{Build-in Self Test}
    \item [BST] \textit{Boundary Scan Test}
    \item [CA] Corrente Alternada
    \item [CC] Corrente Contínua
    \item [DfD] \textit{Design for Debug}
    \item [DfT] \textit{Design for Testability}
    \item [DfY] \textit{Design for Yield}
    \item [EDGE] \textit{Enhanced Data rates for GSM Evolution}
    \item [FPGA] \textit{Field-programmable Gate Array}
    \item [FPT] \textit{Flying Probe Test}
    \item [GPRS] G\textit{eneral Packet Radio Service}
    \item [ICT] \textit{In-Circuit Test}
    \item [IEEE] \textit{Institute of Electrical and Electronics Engineers}
    \item [JTAG] \textit{Joint Test Action Group}
    \item [M2M] \textit{Machine-to-machine}
    \item [MDA] \textit{Manufacturing Defect Analysis}
    \item [NFF] \textit{No failure found}
    \item [NoC] \textit{Network-on-a-Chip}
    \item [OEM] \textit{Original Equipment Manufacturer}
    \item [OTAP] \textit{Over-the-air Provisioning}
    \item [PCI] Placa de circuito Impresso
    \item [PCB] \textit{Printed Circuit Board}
    \item [PCIM] Placa de circuito impresso e montada
    \item [PRPG] \textit{Pseudo-random Pattern Generator}
    \item [QA] \textit{Quality Assurance}
    \item [RSN] \textit{Reconfigurable Scaning Networks}
    \item [SIM] S\textit{ubscriber Identification Module}
    \item [SNR] \textit{Signal-Noise Ratio}
    \item [SPI] \textit{Solder Paste Inspection}
    \item [STUMPS] 
    \item [SoC] \textit{System-on-a-chip}
    \item [VILW] \textit{Very long instruction word}
    
\end{siglas}
% ---

% ---
% inserir lista de símbolos
% ---
%\begin{simbolos}
%  \item[$ \Gamma $] Letra grega Gama
%  \item[$ \Lambda $] Lambda
%  \item[$ \zeta $] Letra grega minúscula zeta
%  \item[$ \in $] Pertence
%\end{simbolos}
% ---l

% ---
% inserir o sumario
% ---
\pdfbookmark[0]{\contentsname}{toc}
\tableofcontents*
\cleardoublepage
% ---


% ----------------------------------------------------------
% ELEMENTOS TEXTUAIS
% ----------------------------------------------------------
\textual
