% ---
% Inserir a ficha bibliografica
% ---

% Isto é um exemplo de Ficha Catalográfica, ou ``Dados internacionais de
% catalogação-na-publicação''. Você pode utilizar este modelo como referência. 
% Porém, provavelmente a biblioteca da sua universidade lhe fornecerá um PDF
% com a ficha catalográfica definitiva após a defesa do trabalho. Quando estiver
% com o documento, salve-o como PDF no diretório do seu projeto e substitua todo
% o conteúdo de implementação deste arquivo pelo comando abaixo:
%

%\imprimirfichacatalografica
\begin{comment}
\begin{fichacatalografica}
	\vspace*{\fill}					% Posição vertical
	\hrule							% Linha horizontal
	\begin{center}					% Minipage Centralizado
	\begin{minipage}[c]{12.5cm}		% Largura
	
	\imprimirautor
	
	\hspace{0.5cm} \imprimirtitulo  / \imprimirautor. --
	\imprimirlocal, \imprimirdata-
	
	\hspace{0.5cm} \pageref{LastPage} p. : il. (algumas color.) ; 30 cm.\\
	
	\hspace{0.5cm} \imprimirorientadorRotulo~\imprimirorientador\\
	
	\hspace{0.5cm}
	\parbox[t]{\textwidth}{\imprimirtipotrabalho~--~\imprimirinstituicao,
	\imprimirdata.}\\
	
	\hspace{0.5cm}
		1. Palavra-chave1.
		2. Palavra-chave2.
		I. Orientador.
		II. Universidade Federal de Santa Catarina.
		III. Departamento de Engenharia Elétrica e Eletrônica.
		IV. Título\\ 			
	
	\hspace{8.75cm} CDU 02:141:005.7\\
	
	\end{minipage}
	\end{center}
	\hrule
\end{fichacatalografica}
\end{comment}
% ---

% ---
% Inserir errata
%

% ---

% ---
% Inserir folha de aprovação
% ---

% Isto é um exemplo de Folha de aprovação, elemento obrigatório da NBR
% 14724/2011 (seção 4.2.1.3). Você pode utilizar este modelo até a aprovação
% do trabalho. Após isso, substitua todo o conteúdo deste arquivo por uma
% imagem da página assinada pela banca com o comando abaixo:
%
%

\begin{comment}
\begin{folhadeaprovacao}

  \begin{center}
    {\ABNTEXchapterfont\large\imprimirautor}

    \vspace*{\fill}\vspace*{\fill}
    \begin{center}
      \ABNTEXchapterfont\bfseries\Large\imprimirtitulo
    \end{center}
    \vspace*{\fill}
    
    \hspace{.45\textwidth}
    \begin{minipage}{.5\textwidth}
        \imprimirpreambulo
    \end{minipage}%
    \vspace*{\fill}
   \end{center}
        
   Trabalho aprovado. \imprimirlocal, 07 de Julho de 2017:

   \assinatura{\textbf{\imprimirorientador, Dr.} \\ Orientador \\ Universidade Federal de Santa Catarina} 
   \assinatura{\textbf{Fabian Leonardo Cabrera Riano, Dr.}  \\ Universidade Federal de Santa Catarina}
   \assinatura{\textbf{Renato Lucas Pacheco, Dr.}  \\ Universidade Federal de Santa Catarina}
   %\assinatura{\textbf{Professor} \\ Convidado 3}
   %\assinatura{\textbf{Professor} \\ Convidado 4}
      
   \begin{center}
    \vspace*{0.5cm}
    {\large\imprimirlocal}
    \par
    {\large\imprimirdata}
    \vspace*{1cm}
  \end{center}
  
\end{folhadeaprovacao}
\end{comment}
% ---

% ---
% Dedicatória
% ---
\begin{dedicatoria}
   \vspace*{\fill}
   \centering
   \noindent
    \textit{Este trabalho é dedicado à minha mãe.} \vspace*{\fill}
\end{dedicatoria}
% ---

% ---
% Agradecimentos
% ---
\begin{agradecimentos}



Agradeço primeiramente ao Professor Fernando Rangel pela paciência e profissionalismo em me orientar neste trabalho. Agradeço à Universidade Federal de Santa Catarina por me proporcionar educação gratuita e de qualidade, não só profissional como também cidadã.

Agradeço à Heloisa pelas muitas revisões textuais e pelo suporte e carinho em todos os momentos. 
Aos meus amigos de graduação pelo companheirismo neste anos de UFSC, em especial ao Lucas Stéfano, por revisar meu trabalho. E por fim, agradeço à Mirian, minha mãe, pelo apoio incondicional.

\end{agradecimentos}
% ---



% ---
% RESUMOS
% ---

% resumo em português
\setlength{\absparsep}{18pt} % ajusta o espaçamento dos parágrafos do resumo
\begin{resumo}
    Este trabalho relata o processo de desenvolvimento de um software de testes funcionais para validação e detecção de falhas em placas eletrônicas em seu processo produtivo assim como nos retornos de manutenção. Os requisitos de flexibilidade e modularidade levaram a escolha do \textit{framework} de atores, a implementação do Modelo de Atores de computação concorrente em LabVIEW. Foram criadas classes de atores para cada área de competência de teste: interface com multímetros, comunicação com o dispositivo sob teste, teste de potência RF, gerador de registros de teste e o controle de execução. Melhorias em ergonomia foram consideradas durante o processo de desenvolvimento, e testes prévios de execução das rotinas de teste apontaram melhorias no tempo de execução de testes e automatização do processo.
    
 
    \textbf{Palavras-chaves}: teste funcional. teste estrutural. modelo de atores. programação concorrente. LabVIEW. montagem de placas de circuito impresso. 
\end{resumo}

% resumo em inglês
\begin{resumo}[Abstract]
 \begin{otherlanguage*}{english}
   
   This work reports the development process of a functional test sofware for fail detection and validation of assembled circuit boards on their productive process and in field returns. The requirements in flexibility and modularity led to the choice of the actor framework, the Labview implementation of the Actor Model of concurrent computing. Actor classes were developed for each field of competence: multimeter interface, device under test communication, RF power test, test log generator, and execution control. Enhancements in ergonomics were considered during the development process, and prior tests runs of the routines indicated improvements in the test execution time and process automation.
   
   \vspace{\onelineskip}
 
   \noindent 
   \textbf{Key-words}: funcional test. structural test. actor model. concurrent programming. LabVIEW. printed circuit board assembly.
 \end{otherlanguage*}
\end{resumo}

% ---
% inserir lista de ilustrações
% ---
\pdfbookmark[0]{\listfigurename}{lof}
\listoffigures*
\cleardoublepage
% ---

% ---
% inserir lista de tabelas
% ---
\pdfbookmark[0]{\listtablename}{lot}
\listoftables*
\cleardoublepage
% ---

% ---
% inserir lista de abreviaturas e siglas
% ---
\begin{siglas}
    \item [AOI] \textit{Automated Optical Inspection}
    \item [ASIC] \textit{Application-Specific Integrated Circuit}
    \item [ATS] \textit{Automated Test Systems}
    \item [AXI] \textit{Automated X-ray Inspection}
    \item [BERT] \textit{Bit-error Rate}
    \item [BISD] \textit{Build-in Self Diagnosis}
    \item [BIST] \textit{Build-in Self Test}
    \item [BST] \textit{Boundary Scan Test}
    \item [CA] Corrente Alternada
    \item [CAN] \textit{Controller Area Network}
    \item [CC] Corrente Contínua
    \item [DfD] \textit{Design for Debug}
    \item [DfT] \textit{Design for Testability}
    \item [DfY] \textit{Design for Yield}
    \item [EDGE] \textit{Enhanced Data rates for GSM Evolution}
    \item [E/S] Entrada/Saída
    \item [FPGA] \textit{Field-programmable Gate Array}
    \item [FPT] \textit{Flying Probe Test}
    \item [GPRS] \textit{General Packet Radio Service}
    \item [ICL] \textit{Instrument Connectivity Language} 
    \item [ICT] Teste Intra-circuito ou \textit{In-Circuit Test}
    \item [IEEE] \textit{Institute of Electrical and Electronics Engineers}
    \item [IMEI] \textit{International Mobile Equipment Identity}
    \item [IP] \textit{intellectual property}
    \item [JSON] \textit{JavaScript Object Notation}
    \item [JTAG] \textit{Joint Test Action Group}
    \item [LAN] \textit{Local Area Network}
    \item [LIN] \textit{Local Interconnect Network}
    \item [LVDS] \textit{Low Voltage Differential signaling}
    \item [M2M] \textit{Machine-to-machine}
    \item [MDA] \textit{Manufacturing Defect Analysis}
    \item [NFF] \textit{No failure found}
    \item [NoC] \textit{Network-on-a-Chip}
    \item [OEM] \textit{Original Equipment Manufacturer}
    \item [OTAP] \textit{Over-the-air Provisioning}
    \item [PCI] Placa de circuito Impresso
    \item [PCIe] \textit{Peripheral Component Interconnect Express}
    \item [PCB] \textit{Printed Circuit Board}
    \item [PCIM] Placa de circuito impresso e montada
    \item [PDL] \textit{procedural description language}
    \item [PRPG] \textit{Pseudo-random Pattern Generator}
    \item [QA] \textit{Quality Assurance}
    \item [RSN] \textit{Reconfigurable Scaning Networks}
    \item [SATA] \textit{Serial AT Attachment}
    \item [SIM] \textit{Subscriber Identification Module}
    \item [SIMcard] \textit{Subscriber Identification Module card}
    \item [SNR] \textit{Signal-Noise Ratio}
    \item [SPI] \textit{Serial Peripheral Interface}
    \item [SoC] \textit{System-on-a-chip}
    \item [SubVI] \textit{Sub Virtual Instrument}
    \item [UART] \textit{Universal Asynchronous Receiver/Transmitter}
    \item [VI] \textit{Virtual Instrument}
    \item [VILW] \textit{Very long instruction word}
    \item [XML] \textit{eXtensible Markup Language}
    
    
\end{siglas}
% ---

% ---
% inserir lista de símbolos
% ---
%\begin{simbolos}
%  \item[$ \Gamma $] Letra grega Gama
%  \item[$ \Lambda $] Lambda
%  \item[$ \zeta $] Letra grega minúscula zeta
%  \item[$ \in $] Pertence
%\end{simbolos}
% ---l

% ---
% inserir o sumario
% ---
\pdfbookmark[0]{\contentsname}{toc}
\tableofcontents*
\cleardoublepage
% ---


% ----------------------------------------------------------
% ELEMENTOS TEXTUAIS
% ----------------------------------------------------------
\textual
